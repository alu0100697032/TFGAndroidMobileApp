%
% ---------------------------------------------------
%
% Trabajo Fin de Grado:
% Author: Víctor Hernández Pérez 
% Correo: alu0100697032@ull.edu.es
% Capítulo: Desarrollo
%
% ----------------------------------------------------
%
\chapter{Desarrollo} \label{chap:Desarrollo}  

No es objeto de esta memoría explicar exhaustivamente cada una de las clases y métodos que componen \App{}.
En este capítulo se describe la implementación de las \textit{Activities} y clases más relevantes de la aplicación. 

\section{Lado del servidor}
Para el desarrollo de la aplicación, en una primera fase, se ha tenido que desplegar un servidor local que soporte 
servicio de autenticación (CAS \cite{URL::CAS}) y simule al de la Universidad de La Laguna. 

\subsection{Servicio CAS}
Dicho servicio CAS \cite{URL::CAS} corre sobre un servidor Apache Tomcat \cite {URL::Tomcat} el cual es una implementación 
de código abierto de Java Servlet \cite{URL::JavaServlet} de Oracle \cite{URL::Oracle}.

\subsection{Protocolos de comunicación}
...
\subsection{Maven}
Maven \cite{URL::Maven} es una herramienta software para la construcción de proyectos Java \cite{URL::Java}...

CAS \cite{URL::CAS} de la Universidad de La Laguna. Dicho 

\section{Clase \textit{Ejemplo}}

Cada instancia de la clase \textit{Ejemplo} ...

\lstinputlisting[float, language=xml, caption={Clase Ejemplo}, label={code:ejemplo}]
{listings/ejemplo.java} %% LISTING
