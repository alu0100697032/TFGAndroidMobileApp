%
% ---------------------------------------------------
%
% Trabajo Fin de Grado:
% Author: Víctor Hernández Pérez 
% Correo: alu0100697032@ull.edu.es
% Capítulo: Herramientas Software
%
% ----------------------------------------------------
%

\cleardoublepage
\chapter{Herramientas Software de desarrollo en Android} \label{chap:polytopes}  

En este capítulo se describen las herramientas software necesarias para la elaboración del TFG.

\section{Android Studio}

Android Studio \cite{URL::AndroidStudio} es un nuevo entorno de desarrollo integrado para el sistema operativo Android comercializado por Google, 
diseñado para ofrecer nuevas herramientas para el desarrollo de aplicaciones y alternativa al entorno Eclipse \cite{URL::Eclipse}, 
hasta ahora el IDE más utilizado.

¿Qué ofrece Android Studio? 
\begin{itemize}
\item Un entorno de desarrollo claro y robusto.
\item Facilidad para testear el funcionamiento en diversos tipos de dispositivos.
\item Asistentes y plantillas para los elementos comunes de programación en Android.
\item Un completo editor con muchas herramientas extra para agilizar el desarrollo de nuestras aplicaciones.
\end{itemize}

\subsection{Instalación}
Para instalar Android Studio es necesario disponer del Java Development Kit (JDK) \cite{URL::JDKInfo}. 

Una vez completada la instalación del JDK, se procede a la descarga del AndroidStudio \cite{URL::AndroidStudio}, del SDK de Android
\cite{URL::InstallSDK} y de todos los paquetes del SDK \cite{URL::SDKPackages} necesarios para asegurar la compatibilidad con los dispositivos Android en
los que se desee desarrollar.

\subsection{Primeros pasos en Android Studio}

Antes de comenzar con el desarrollo de la aplicación propuesta para el TFG, se realizaron una serie de tutoriales \cite{URL::GettingStarted, URL::SavingData} para familiarizarse con el uso de AndroidStudio así como de todo el entorno Android.

\section{Jasig}

Jasig \cite{URL::Jasig} es un Servicio de Autenticación Centralizada, Central Authentication Service (CAS \cite{URL::CAS}) en inglés.  
Está orientado a aplicaciones web y soporta multitud de protocolos de comunicación entre el servidor y el cliente así como: 
CAS, SAML, Oauth \cite{URL::Oauth} y OpenID \cite{URL::OpenID}.

\subsection{Instalación}
...


