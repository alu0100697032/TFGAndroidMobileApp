%
% ---------------------------------------------------
%
% Trabajo Fin de Grado:
% Author: Víctor Hernández Pérez 
% Correo: alu0100697032@ull.edu.es
%
% ----------------------------------------------------
%
\documentclass[spanish,a4paper,14pt,oneside]{extreport}
%\documentclass[a4paper, twoside, 12pt]{book}
\usepackage[a4paper]{geometry}
\usepackage[spanish]{babel}
\usepackage[utf8]{inputenc}
%\usepackage{lscape}
\usepackage{pdflscape}
%%%%%%%%%%%%%%%%%%%%%%%%%%%%%%%%%%%%%%%%%%%%%%%%%%%%%%%%%%%%%%%%%%%%%%%%%%%%%%%%%%%%%%%%%%%%
% Next 3+3 lines select PDF or PS output (comment as apropriate)
% To switch from PDF and PS comment/uncomment here and change Makefile
\usepackage[pdftex]{color}
\usepackage[pdftex]{graphicx}
\usepackage{float}
\usepackage{eurosym}
\graphicspath{{images/}}
%\usepackage[dvips]{color}
%\usepackage[dvips]{graphicx}
\usepackage{epsfig}
%\graphicspath{{images/eps/}} 
%%%%%%%%%%%%%%%%%%%%%%%%%%%%%%%%%%%%%%%%%%%%%%%%%%%%%%%%%%%%%%%%%%%%%%%%%%%%%%%%%%%%%%%%%%%%
\usepackage{algorithmic}
\usepackage[pdftex=true,colorlinks=false,urlcolor=blue,plainpages=false,pagebackref=true,citecolor=red]{hyperref} %hiperenlaces y backcites 
%%%%%%%%%%%%%%%%%%%%%%%%%%%%%%%%%%%%%%%%%%%%%%%%%%%%%%%%%%%%%%%%%%%%%%%%%%%%%%%%%%%%%%%%%%%%
% Comandos para escribir "siempre igual"
\newcommand{\App}{\texttt{UllApp{}}}
\newcommand{\lApp}{\texttt{Aplicación Android para la Universidad de La Laguna{}}}

%%% Traducimos el pseudocodigo
\renewcommand{\algorithmicwhile}{\textbf{mientras}}
\renewcommand{\algorithmicend}{\textbf{fin}}
\renewcommand{\algorithmicdo}{\textbf{hacer}}
\renewcommand{\algorithmicif}{\textbf{si}}
\renewcommand{\algorithmicthen}{\textbf{entonces}}
\renewcommand{\algorithmicrepeat}{\textbf{repetir}}
\renewcommand{\algorithmicuntil}{\textbf{hasta que}}
\renewcommand{\algorithmicelse}{\textbf{en otro caso}}
\renewcommand{\algorithmicfor}{\textbf{para}}

%%%%%%%%%%%%%%%%% Creamos un entorno para listar código fuente %%%%%%%%%%%%%%%
\newenvironment{sourcecode}
{\begin{list}{}{\setlength{\leftmargin}{1em}}\item\scriptsize\bfseries}
{\end{list}}

\newenvironment{littlesourcecode}
{\begin{list}{}{\setlength{\leftmargin}{1em}}\item\tiny\bfseries}
{\end{list}}

\newenvironment{summary}
{\par\noindent\begin{center}\textbf{Abstract}\end{center}\begin{itshape}\par\noindent}
{\end{itshape}}

\newenvironment{keywords}
{\begin{list}{}{\setlength{\leftmargin}{1em}}\item[\hskip\labelsep \bfseries Keywords:]}
{\end{list}}

\newenvironment{palabrasClave}
{\begin{list}{}{\setlength{\leftmargin}{1em}}\item[\hskip\labelsep \bfseries Palabras clave:]}
{\end{list}}


%%%%%%%%%%%%%%%%%%%%%%%%%%%%%%%%%%%%%%%%%%%%%%%%%%%%%%%%%%%%%%%%%%%%%%%%%%%%%%%
\definecolor{marron}       {rgb}{0.496, 0.203, 0.152}
\definecolor{verde-claro}  {rgb}{0.625, 0.734, 0.199}
\definecolor{oscuro}       {rgb}{0.187, 0.141, 0.285}
\definecolor{gris}     	   {rgb}{0.500, 0.500, 0.500}
\definecolor{bgd-listings} {rgb}{0.999, 0.999, 0.900}
\definecolor{gray97}{gray}{.97}
\definecolor{gray75}{gray}{.75}
\definecolor{gray45}{gray}{.45}
\definecolor{gray}{gray}{.45}
%%%%%%%%%%%%%%%%%%%%%%%%%%%%%%%%%%%%%%%%%%%%%%%%%%%%%%%%%%%%%%%%%%%%%%%%%%%%%%%%%%%%%%%%%%%%
%%% Code Listings
%\usepackage{listings} 
%\lstloadlanguages{python,C}
\definecolor{Brown}{cmyk}{0,0.81,1,0.60}
\definecolor{OliveGreen}{cmyk}{0.64,0,0.95,0.40}
\definecolor{CadetBlue}{cmyk}{0.62,0.57,0.23,0}
\definecolor{lightlightgray}{gray}{0.9}
%%%%%%%%%%%%%%%%%%%%%%%%%%%%%%%%%%%%%%%%%%%%%%%%%%%%%%%%%%%%%%%%%%%%%%%%%%%%%%%%%%%%%%%%%%%
%Evitar partir palabras al final de la línea
%\hyphenpenalty=10000
%\tolerance=1000
%%%%%%%%%%%%%%%%%%%%%%%%%%%%%%%%%%%%%%%%%%%%%%%%%%%%%%%%%%%%%%%%%%%%%%%%%%%%%%%%%%%%%%%%%%%%
% Para listados de código
\usepackage{listings}
\lstloadlanguages{C}

% Definiendo colores para los listados de código fuente - Univ. Deusto
\definecolor{violet}{rgb}{0.5,0,0.5}
\definecolor{navy}{rgb}{0,0,0.5}
\definecolor{hellgelb}{rgb}{1,1,0.8}
\definecolor{colKeys}{rgb}{0,0,1}
\definecolor{colIdentifier}{rgb}{0,0,0}
\definecolor{colComments}{rgb}{1,0,0}
\definecolor{colString}{rgb}{0,0.5,0}

%\lstset{morekeywords={pragma copy\_in copy\_out copy omp parallel private reduction shared hicuda loop\_partition over\_tblock over\_thread}}
\lstset{
        float=tbhp,
		    language = Java,
				morekeywords={llc,reduction_type,nc_result,
				              hicuda,global,alloc,shape,kernel,thread,loop_partition,tblock,over_tblock,over_thread,kernel_end,copyout,free,
											data,region,
											task,input,inout,output,
				              pragma,omp,parallel,reduction,private,shared,target,device,copy_in,copy_out,
				              acc,kernels,loop,copyin,copy,pcopy,pcopyin,collapse,gang,worker,independent},
				%\emph      ={omp,parallel,reduction,private,shared},
				emphstyle=\textbf,
        %basicstyle=\ttfamily\tiny,
        basicstyle=\ttfamily\scriptsize,
        identifierstyle=\color{colIdentifier},
        keywordstyle=\color{colKeys},
        stringstyle=\color{colString},
        commentstyle=\color[rgb]{0.133,0.545,0.133},
        columns=flexible,
        tabsize=4,
        frame=single,
        extendedchars=true,
        showspaces=false,
        showstringspaces=false,
        numbers=left,
        numberstyle=\tiny,
        breaklines=true,
        backgroundcolor=\color{lightlightgray},
        breakautoindent=true,
        captionpos=b
}

%\renewcommand{\lstlistingname}{Listing} % Los títulos de los códigos insertados se denotan con Ejemplo...   

% Otro formato más bonito para código fuente
\newcommand{\codigofuente}[3]{%
  \lstlisting[language=#1,caption={#2}]{#3}%
}
%%%%%%%%%%%%%%%%%%%%%%%%%%%%%%%%%%%%%%%%%%%%%%%%%%%%%%%%%%%%%%%%%%%%%%%%%%%%%%%
\begin{document}
\renewcommand{\lstlistingname}{Listado}% Listing -> Listado de código
%%%%%%%%%%%%%%%%%%%%%%%%%%%%%%%%%%%%%%%%%%%%%%%%%%%%%%%%%%%%%%%%%%%%%%%%%%%%%%%
% First Page
%%%%%%%%%%%%%%%%%%%%%%%%%%%%%%%%%%%%%%%%%%%%%%%%%%%%%%%%%%%%%%%%%%%%%%%%%%%%%%%

\pagestyle{empty}
\thispagestyle{empty}


\newcommand{\HRule}{\rule{\linewidth}{1mm}}
\setlength{\parindent}{0mm}
\setlength{\parskip}{0mm}

\vspace*{\stretch{0.5}}

\begin{center}
\includegraphics[scale=0.8]{images/logo_vertical}\\[10mm]
{\Huge Trabajo de Fin de Grado}
\end{center}

\HRule
\begin{flushright}
        {\Huge \lApp{}} \\[2.5mm]
        {\Large \textit{\App}} \\[5mm]
        {\Large Víctor Hernández Pérez} \\[5mm]


\end{flushright}
\HRule
\vspace*{\stretch{2}}
\begin{center}
  \Large La Laguna, \today
\end{center}

\setlength{\parindent}{5mm}

%%%%%%%%%%%%%%%%%%%%%%%%%%%%%%%%%%%%%%%%%%%%%%%%%%%%%%%%%%%%%%%%%%%%%%%%%%%%%%%
% Signature page (add the official stamp)
%%%%%%%%%%%%%%%%%%%%%%%%%%%%%%%%%%%%%%%%%%%%%%%%%%%%%%%%%%%%%%%%%%%%%%%%%%%%%%%
\newpage
%\cleardoublepage
\thispagestyle{empty}

D. {\bf Francisco de Sande González}, con D.N.I. 42.067.050-G
profesor
Titular de Universidad
adscrito al Departamento
de Ingeniería Informática y de Sistemas
de la Universidad de La Laguna, como tutor

\bigskip

\bigskip
\bigskip
{\bf C E R T I F I C A}

\bigskip
\bigskip
\bigskip
Que la presente memoria titulada:

\bigskip
``{\it \lApp{}}''

\bigskip
\bigskip
\bigskip

\noindent ha sido realizada bajo su dirección por D. {\bf Víctor Hernández Pérez},
con D.N.I. 78.643.409-S

\bigskip
\bigskip

Y para que así conste, en cumplimiento de la legislación vigente y a los efectos
oportunos firman la presente en La Laguna a \today

%\cleardoublepage
\newpage
%%%%%%%%%%%%%%%%%%%%%%%%%%%%%%%%%%%%%%%%%%%%%%%%%%%%%%%%%%%%%%%%%%%%%%%%%%%%%%%
\thispagestyle{empty}

{ \flushright

\begin{LARGE}
Agradecimientos
\end{LARGE}

\hspace{3mm}

\begin{large}


\hspace{3mm}
En primer lugar agradecer a mi familia por haberme educado, apoyado y haber hecho el esfuerzo que me ha permitido realizar estos estudios.

Dar las gracias también a mi tutor, Francisco de Sande González por su orientación, predisposición, conocimiento y experiencia que me han ayudado en la realización de éste trabajo.

Y por último, agradecer al resto de profesores y compañeros que he tenido a lo largo de la carrera, porque de todos he aprendido.


\end{large}

}

%%%%%%%%%%%%%%%%%%%%%%%%%%%%%%%%%%%%%%%%%%%%%%%%%%%%%%%%%%%%%%%%%%%%%%%%%%%%%%%%%
\newpage

\begin{huge}
Licencia
\end{huge}

\bigskip
%* Si quiere permitir que se compartan las adaptaciones de tu obra mientras se comparta de la misma manera
%y NO quieres permitir usos comerciales de tu obra indica:

\begin{center}
\includegraphics[scale=1.5]{images/by-nc-sa_88x31}\\[10mm]
{\Large \copyright~Esta obra está bajo una licencia de Creative Commons Reconocimiento-NoComercial-CompartirIgual 4.0 Internacional.
}
\end{center}


%%%%%%%%%%%%%%%%%%%%%%%%%%%%%%%%%%%%%%%%%%%%%%%%%%%%%%%%%%%%%%%%%%%%%%%%%%%%%%%
\newpage  %\cleardoublepage
\begin{abstract}
{\em

El objetivo de este trabajo ha sido crear una aplicación para dispositivos Android que permita autenticación al usuario, acceder a algún tipo de servicio y a otros datos de interés relacionados con la Universidad de La Laguna.

\bigskip
Partiendo de los conocimientos en \textit{Java} obtenidos en la asignatura: \textit{'Programación de Aplicaciones Interactivas'} y desarrollados en otras asignaturas como: 
\textit{'Diseño y Análisis de Algoritmos'}, \textit{'Complejidad Computacional'} o \textit{'Inteligencia Artificial'} impartidas, la mayor parte de las mismas, en el 
itinerario de computación del Grado en Ingeniería Informática de la Universidad de La Laguna, se espera adquirir los 
conocimientos básicos necesarios para desarrollar aplicaciones en Android. 

\bigskip
Principalmente la aplicación permite al usuario hacer login con su cuenta de Facebook o Google para poder acceder a las noticias institucionales, ubicaciones de interés en Google Maps y hacer uso del sistema de un sistema de reserva de pistas deportivas.

\bigskip 
Del mismo modo, haciendo uso de algunos conocimientos adquiridos en la asignatura \textit{Prácticas Externas}, se ha conseguido implementar otra aplicación en Django que sea la encargada de controlar a los usuarios autenticados y haga la gestión del sistema de reservas.

}
\begin{palabrasClave}
Aplicaciones Android, Django, autenticación y Google Maps.
\end{palabrasClave}

\end{abstract}
%%%%%%%%%%%%%%%%%%%%%%%%%%%%%%%%%%%%%%%%%%%%%%%%%%%%%%%%%%%%%%%%%%%%%%%%%%%%%%%

%%%%%%%%%%%%%%%%%%%%%%%%%%%%%%%%%%%%%%%%%%%%%%%%%%%%%%%%%%%%%%%%%%%%%%%%%%%%%%%
\newpage  %\cleardoublepage
\begin{summary}
{\em

The objective of this work has been the development of an application for Android devices that provides authentication, access to a service and access other data of interest related with University of La Laguna.

\bigskip
Based on the knowledge of Java obtained in the subject: ``Programming Interactive Applications'' and developed in other subjects like: ``Design and Analysis of Algorithms'', 
``Computational Complexity'' or``Artificial Intelligence''  taken, most of them, in the
Computer itinerary in the third year of the Universidad de La Laguna Degree in Computing Engineering, 
in this work we expect to acquire the basic knowledge needed to
develop Android applications.

\bigskip
Mainly, the application allows the user login with his Facebook or Google account and get institutional news, location of interest on Google Maps and use the reserve system of sports fields.

\bigskip
Moreover, making use of some knowledge of Django acquired in the subject ``Prácticas Externas'', it has beed achieved develop another Django application which handdle the authtenticated users and manage the reserve system. 

}

\begin{keywords}
Application for Android, Django, authentication and Google Maps.
\end{keywords}

\end{summary}
%%%%%%%%%%%%%%%%%%%%%%%%%%%%%%%%%%%%%%%%%%%%%%%%%%%%%%%%%%%%%%%%%%%%%%%%%%%%%%%

%%%%%%%%%%%%%%%%%%%%%%%%%%%%%%%%%%%%%%%%%%%%%%%%%%%%%%%%%%%%%%%%%%%%%%%%%%%%%%%
\newpage{\pagestyle{empty}}
\thispagestyle{empty}

%%%%%%%%%%%%%%%%%%%%%%%%%%%%%%%%%%%%%%%%%%%%%%%%%%%%%%%%%%%%%%%%%%%%%%%%%%%%%%%


\pagestyle{myheadings} %my head defined by markboth or markright
% No funciona bien \markboth sin "twoside" en \documentclass, pero al
% ponerlo se dan un montón de errores de underfull \vbox, con lo que no se
% ha puesto.
\markboth{Víctor Hernández Pérez}{\App{}}

%%%%%%%%%%%%%%%%%%%%%%%%%%%%%%%%%%%%%%%%%%%%%%%%%%%%%%%%%%%%%%%%%%%%%%%%%%%%%%%
%Numeracion en romanos
\renewcommand{\thepage}{\roman{page}}
\setcounter{page}{1}

%%%%%%%%%%%%%%%%%%%%%%%%%%%%%%%%%%%%%%%%%%%%%%%%%%%%%%%%%%%%%%%%%%%%%%%%%%%%%%%

\tableofcontents

%%%%%%%%%%%%%%%%%%%%%%%%%%%%%%%%%%%%%%%%%%%%%%%%%%%%%%%%%%%%%%%%%%%%%%%%%%%%%%%
\newpage{\pagestyle{empty}}

\listoffigures

%%%%%%%%%%%%%%%%%%%%%%%%%%%%%%%%%%%%%%%%%%%%%%%%%%%%%%%%%%%%%%%%%%%%%%%%%%%%%%%
\newpage{\pagestyle{empty}}

%\listoftables

%%%%%%%%%%%%%%%%%%%%%%%%%%%%%%%%%%%%%%%%%%%%%%%%%%%%%%%%%%%%%%%%%%%%%%%%%%%%%%%
\newpage{\pagestyle{empty}}

%%%%%%%%%%%%%%%%%%%%%%%%%%%%%%%%%%%%%%%%%%%%%%%%%%%%%%%%%%%%%%%%%%%%%%%%%%%%%%%
%Numeracion a partir del capitulo I
\renewcommand{\thepage}{\arabic{page}}
\setcounter{page}{1}


%%%%%%%%%%%%%%%%%%%%%%%%%%%%%%%%%%%%%%%%%%%%%%%%%%%%%%%%%%%%%%%%%%%%%%%%%%%%%%%
%\chapter{Introducción}
%\label{chapter:intro}
%\input{cap1.tex}
%%%%%%%%%%%%%%%%%%%%%%%%%%%%%%%%%%%%%%%%%%%%%%%%%%%%%%%%%%%%%%%%%%%%%%%%%%%%%%%
%\chapter{Título del Capítulo Dos}
%\label{chapter:dos}
%\input{cap2.tex}
%%%%%%%%%%%%%%%%%%%%%%%%%%%%%%%%%%%%%%%%%%%%%%%%%%%%%%%%%%%%%%%%%%%%%%%%%%%%%%%%
%\newpage{\pagestyle{empty}}
%\thispagestyle{empty}
%\chapter{Título del Capítulo Tres}
%\label{chapter:tres}
%\input{cap3.tex}
%%%%%%%%%%%%%%%%%%%%%%%%%%%%%%%%%%%%%%%%%%%%%%%%%%%%%%%%%%%%%%%%%%%%%%%%%%%%%%%%
%\chapter{Título del Capítulo Cuatro}
%\label{chapter:cuatro}
%\input{cap4.tex}
%%%%%%%%%%%%%%%%%%%%%%%%%%%%%%%%%%%%%%%%%%%%%%%%%%%%%%%%%%%%%%%%%%%%%%%%%%%%%%%%
%\newpage{\pagestyle{empty}}
%\thispagestyle{empty}
%\chapter{Conclusiones y líneas futuras}
%\label{chapter:Conclusiones}
%\input{cap5.tex}
%%%%%%%%%%%%%%%%%%%%%%%%%%%%%%%%%%%%%%%%%%%%%%%%%%%%%%%%%%%%%%%%%%%%%%%%%%%%%%%%
%\newpage{\pagestyle{empty}}
%\thispagestyle{empty}
%\chapter{Summary and Conclusions }
%\label{chapter:ingles}
%\input{cap6.tex}
% ==========================================================
% --------               Capítulos                ----------
% --------    Estan en el directorio capitulos/   ----------
% ==========================================================
%
% ---------------------------------------------------
%
% Trabajo Fin de Grado:
% Author: Víctor Hernández Pérez 
% Correo: alu0100697032@ull.edu.es
% Capítulo: Prólogo
%
% ----------------------------------------------------
%
\chapter*{Introducción}
\addcontentsline{toc}{chapter}{Introducción} 

En este documento se recoge el trabajo de investigación y desarrollo realizado por el 
autor para completar su Trabajo de Fin de Grado (TFG), finalizando con el mismo
los estudios en el Grado en Ingeniería en Informática cursados en la Escuela Superior 
de Ingerniería y Tecnología - Sección Informática (ESIT-Informática) de la ULL.


%
% ---------------------------------------------------
%
% Trabajo Fin de Grado:
% Author: Víctor Hernández Pérez 
% Correo: alu0100697032@ull.edu.es
% Capítulo: Objetivos
%
% ----------------------------------------------------
%

\chapter{Objetivos} \label{chap:objetives}  

El presente TFG ...

\begin{itemize}
\item Adquirir conocimientos básicos sobre conceptos, modelos, técnicas y herramientas asociadas con la programación de aplicaciones en Android.
\item Puesta en práctica de los conocimientos del Grado en Ingeniería Informática.
\item Creación de una memoria técnica sobre la aplicación desarrollada en el Trabajo de Fin de Grado.
\item Adquirir conocimientos sobre \textit{LaTeX} \cite{URL::LaTeX}, un sistema de composición de textos, orientado a la creación de documentos escritos 
que presenten una alta calidad tipográfica.
\end{itemize}

%
% ---------------------------------------------------
%
% Trabajo Fin de Grado:
% Author: Víctor Hernández Pérez 
% Correo: alu0100697032@ull.edu.es
% Capítulo: Herramientas Software
%
% ----------------------------------------------------
%

\cleardoublepage
\chapter{Herramientas Software de desarrollo en Android} \label{chap:polytopes}  

En este capítulo se describen las herramientas software necesarias para la elaboración del TFG.

\section{Android Studio}

Android Studio \cite{URL::AndroidStudio} es un nuevo entorno de desarrollo integrado para el sistema operativo Android comercializado por Google, 
diseñado para ofrecer nuevas herramientas para el desarrollo de aplicaciones y alternativa al entorno Eclipse \cite{URL::Eclipse}, 
hasta ahora el IDE más utilizado.

¿Qué ofrece Android Studio? 
\begin{itemize}
\item Un entorno de desarrollo claro y robusto.
\item Facilidad para testear el funcionamiento en diversos tipos de dispositivos.
\item Asistentes y plantillas para los elementos comunes de programación en Android.
\item Un completo editor con muchas herramientas extra para agilizar el desarrollo de nuestras aplicaciones.
\end{itemize}

\subsection{Instalación}
Para instalar Android Studio es necesario disponer del Java Development Kit (JDK) \cite{URL::JDKInfo}. 

Una vez completada la instalación del JDK, se procede a la descarga del AndroidStudio \cite{URL::AndroidStudio}, del SDK de Android
\cite{URL::InstallSDK} y de todos los paquetes del SDK \cite{URL::SDKPackages} necesarios para asegurar la compatibilidad con los dispositivos Android en
los que se desee desarrollar.

\subsection{Primeros pasos en Android Studio}

Antes de comenzar con el desarrollo de la aplicación propuesta para el TFG, se realizaron una serie de tutoriales \cite{URL::GettingStarted, URL::SavingData} para familiarizarse con el uso de AndroidStudio así como de todo el entorno Android.

\section{Jasig}

Jasig \cite{URL::Jasig} es un Servicio de Autenticación Centralizada, Central Authentication Service (CAS \cite{URL::CAS}) en inglés.  
Está orientado a aplicaciones web y soporta multitud de protocolos de comunicación entre el servidor y el cliente así como: 
CAS, SAML, Oauth \cite{URL::Oauth} y OpenID \cite{URL::OpenID}.

\subsection{Instalación}
...



%
% ---------------------------------------------------
%
% Trabajo Fin de Grado:
% Author: Víctor Hernández Pérez 
% Correo: alu0100697032@ull.edu.es
% Capítulo: Descripción de la App
%
% ----------------------------------------------------
%

\cleardoublepage

\chapter{Descripción de la aplicación}

\section{Introducción}

\App{} es una aplicación para dispositivos Android (App) que permite ...

\subsection{Subsección}

\begin{figure}[h]
	\centering
	\includegraphics[width=0.3\columnwidth]{cc.png}
	\caption{Imagen de ejemplo}
	\label{fig:ejemplo}
\end{figure}

%%%%%%%%%%%%%%%%%%%%%%%%%%%%%%%%%%%%%%%%
\section{Actividades}

Actividades

%
% ---------------------------------------------------
%
% Trabajo Fin de Grado:
% Author: Víctor Hernández Pérez 
% Correo: alu0100697032@ull.edu.es
% Capítulo: Desarrollo
%
% ----------------------------------------------------
%
\chapter{Desarrollo} \label{chap:Desarrollo}  

No es objeto de esta memoría explicar exhaustivamente cada una de las clases y métodos que componen \App{}.
En este capítulo se describe la implementación de las \textit{Activities} y clases más relevantes de la aplicación. 

\section{Lado del servidor}
Para el desarrollo de la aplicación, en una primera fase, se ha tenido que desplegar un servidor local que soporte 
servicio de autenticación (CAS \cite{URL::CAS}) y simule al de la Universidad de La Laguna. 

\subsection{Servicio CAS}
Dicho servicio CAS \cite{URL::CAS} corre sobre un servidor Apache Tomcat \cite {URL::Tomcat} el cual es una implementación 
de código abierto de Java Servlet \cite{URL::JavaServlet} de Oracle \cite{URL::Oracle}.

\subsection{Protocolos de comunicación}
...
\subsection{Maven}
Maven \cite{URL::Maven} es una herramienta software para la construcción de proyectos Java \cite{URL::Java}...

CAS \cite{URL::CAS} de la Universidad de La Laguna. Dicho 

\section{Clase \textit{Ejemplo}}

Cada instancia de la clase \textit{Ejemplo} ...

\lstinputlisting[float, language=xml, caption={Clase Ejemplo}, label={code:ejemplo}]
{listings/ejemplo.java} %% LISTING

%
% ---------------------------------------------------
%
% Trabajo Fin de Grado:
% Author: Víctor Hernández Pérez 
% Correo: alu0100697032@ull.edu.es
% Capítulo: Despliegue
%
% ----------------------------------------------------
%
\chapter{Despliegue de la aplicación}


%
% ---------------------------------------------------
%
% Trabajo Fin de Grado:
% Author: Víctor Hernández Pérez 
% Correo: alu0100697032@ull.edu.es
% Capítulo: Conclusiones y líneas futuras
%
% ----------------------------------------------------
%
\chapter{Conclusiones y líneas futuras de trabajo} \label{chap:to-do} 

Al concluir el TFG, se han adquirido los conocimientos mínimos tanto para desarrollar aplicaciones para dispositivos Android como aplicaciones Django.

La parte en la que más se ha invertido tiempo ha sido la de investigar las tecnologías usadas, ya que la aplicación cuenta con muchos detalles hasta ahora no conocidos por el autor.

Se ha diseñado, desarrollado y desplegado una aplicación que es capaz de autenticarse, acceder a datos remotos e interactuar con una aplicación Django que oferta un sistema de reservas a los usuarios autenticados. Además se ha integrado la aplicación con Google Maps para dar información acerca de ciertas ubicaciones de interés.

La gestión del tiempo ha sido clave en el desarrollo, ya que se ha compaginado con las asignaturas: \textit{Sistemas Electrónicos Digitales}, \textit{Diseño y Análisis de Algoritmos}, \textit{Inteligencia Emocional} y \textit{Prácticas Externas}. Gracias a ello, se ha obtenido experiencia en la gestión de tareas relacionadas con la Ingeniería del 
Software. 

También se ha trabajado en la elaboración de una memoría técnica con \textit{LaTeX}, desconocido hasta el momento. 

El desarrollo de \App continuará implementando las siguientes funcionalidades:

\begin{itemize}
\item Posibilidad de autenticarse contra los servidores de la ULL.
\item Mejora del servicio de reservas.
\item Integración de más servicios ofertados en la ULL, como por ejemplo la Biblioteca.
\end{itemize}



%
% ---------------------------------------------------
%
% Trabajo Fin de Grado:
% Author: Víctor Hernández Pérez 
% Correo: alu0100697032@ull.edu.es
% Capítulo: Summary and Conlusions
%
% ----------------------------------------------------
%
\chapter{Summary and Conlusions}
At the conclusion of the work, I have acquired the minimum knowledge needed
to develop applications for Android devices.
I have designed, developed and desployed an application that is able to ...

Time management has been very important in this project, because it has combined with the subject ``Prácticas Externas'' also studied in the fourth year 
of the Degree in Computing Engineering. 
As a result, I have improved my skills in managing tasks related to software engineering.

I have also learned to draw up a Technical Report using \textit{LaTeX}, unknown so far.

The development of App could be continued by implementing some of the following features:
\begin{itemize}
\item Feature 1.
\item Feature 2
\end{itemize}
%
% ---------------------------------------------------
%
% Trabajo Fin de Grado:
% Author: Víctor Hernández Pérez 
% Correo: alu0100697032@ull.edu.es
% Capítulo: Presupuesto
%
% ----------------------------------------------------
%
\chapter{Presupuesto}

En esta sección de mostrará una tabla con las actividades principales realizadas, indicando cuanto tiempo aproximado se ha dedicado a cada una y un coste asociado por hora de trabajo del desarrollador. 

Se ha supuesto un sueldo de 20 euros la hora en el que el desarrollo completo de la aplicación supondría dos meses de trabajo. Gastos de despliegue de la aplicación a parte.

\bigskip
% Please add the following required packages to your document preamble:
% \usepackage{graphicx}
\begin{table}[H]
\centering
\resizebox{\textwidth}{!}{%
\begin{tabular}{|l|c|c|}
\hline
\multicolumn{1}{|c|}{\textbf{Actividad}} & \textbf{Duración (horas)} & \textbf{Coste (\euro{})} \\ \hline
Documentación                            & 100                       & 2000                \\ \hline
Diseño                                   & 40                        & 800                \\ \hline
Implementación                           & 80                        & 1600               \\ \hline
Evaluación y pruebas                     & 20                        & 400                \\ \hline
Elaboración de la memoria                & 50                        & 1000               \\ \hline
Revisión                                 & 10                        & 200               \\ \hline
\multicolumn{1}{c}{\textbf{}} & \multicolumn{1}{c}{\textbf{}}  & \multicolumn{1}{c}{\textbf{}} \\ \hline
\multicolumn{1}{|c|}{\textbf{TOTAL}}     & \textbf{300}              & \textbf{6000}      \\ \hline
\end{tabular}%
}
\caption{Tiempo dedicado y coste}
\label{my-label}
\end{table}

%\newpage{\pagestyle{empty}}
%\thispagestyle{empty}

%\chapter{Presupuesto}
%\label{chapter:Presupuesto}

%\input{cap7.tex}

%%%%%%%%%%%%%%%%%%%%%%%%%%%%%%%%%%%%%%%%%%%%%%%%%%%%%%%%%%%%%%%%%%%%%%%%%%%%%%%

%%%%%%%%%%%%%%%%%%%%%%%%%%%%%%%%%%%%%%%%%%%%%%%%%%%%%%%%%%%%%%%%%%%%%%%%%%%%%%%
%\newpage{\pagestyle{empty}}
%\thispagestyle{empty}
%\begin{appendix}
%
%\chapter{Título del Apéndice 1}
%\label{appendix:1}
%\input{apendice1.tex}
%
%\chapter{Título del Apéndice 2}
%\label{appendix:2}
%\input{apendice2.tex}
%
%\end{appendix}
%%%%%%%%%%%%%%%%%%%%%%%%%%%%%%%%%%%%%%%%%%%%%%%%%%%%%%%%%%%%%%%%%%%%%%%%%%%%%%%
\addcontentsline{toc}{chapter}{Bibliografía}
\bibliographystyle{plain}
\renewcommand{\bibname}{Bibliografía}   %  Para que no aparezca Índice de figuras
\bibliography{bibliography}

%%%%%%%%%%%%%%%%%%%%%%%%%%%%%%%%%%%%%%%%%%%%%%%%%%%%%%%%%%%%%%%%%%%%%%%%%%%%%%%

\end{document}
