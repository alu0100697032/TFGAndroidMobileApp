%
% ---------------------------------------------------
%
% Trabajo Fin de Grado:
% Author: Víctor Hernández Pérez 
% Correo: alu0100697032@ull.edu.es
% Capítulo: Conclusiones y líneas futuras
%
% ----------------------------------------------------
%
\chapter{Conclusiones y líneas futuras de trabajo} \label{chap:to-do} 

Al concluir el TFG, se han adquirido los conocimientos mínimos tanto para desarrollar aplicaciones para dispositivos Android como aplicaciones Django.

La parte en la que más se ha invertido tiempo ha sido la de investigar las tecnologías usadas, ya que la aplicación cuenta con muchos detalles hasta ahora no conocidos por el autor.

Se ha diseñado, desarrollado y desplegado una aplicación que es capaz de autenticarse, acceder a datos remotos e interactuar con una aplicación Django que oferta un sistema de reservas a los usuarios autenticados. Además se ha integrado la aplicación con Google Maps para dar información acerca de ciertas ubicaciones de interés.

La gestión del tiempo ha sido clave en el desarrollo, ya que se ha compaginado con las asignaturas: \textit{Sistemas Electrónicos Digitales}, \textit{Diseño y Análisis de Algoritmos}, \textit{Inteligencia Emocional} y \textit{Prácticas Externas}. Gracias a ello, se ha obtenido experiencia en la gestión de tareas relacionadas con la Ingeniería del 
Software. 

También se ha trabajado en la elaboración de una memoría técnica con \textit{LaTeX}, desconocido hasta el momento. 

El desarrollo de \App continuará implementando las siguientes funcionalidades:

\begin{itemize}
\item Posibilidad de autenticarse contra los servidores de la ULL.
\item Mejora del servicio de reservas.
\item Integración de más servicios ofertados en la ULL, como por ejemplo la Biblioteca.
\end{itemize}


